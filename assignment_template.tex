\documentclass[11pt, letterpaper, reqno]{article}
\usepackage[utf8]{inputenc}
\usepackage[english]{babel}
\usepackage{ amssymb }
\usepackage{ mathrsfs }
\pagenumbering{gobble}

\usepackage{titlesec}
\titleformat{\subsection}[runin]{\bfseries}{}{}{}[]
\setlength{\parindent}{0mm}
\usepackage[margin=1.1in]{geometry}

\usepackage{amsthm}
\usepackage{array}
\usepackage{amsmath}
\usepackage{amsfonts} 
\usepackage{tikz}
\usepackage{mathtools}  
\usepackage{physics}
\usepackage{enumitem}
\usepackage{float}
\usepackage{mdframed}
\allowdisplaybreaks % doesn't push the whole align to next page

\title{{course code}: {course name} -- Assignment \# {Assignment \# } }

\author{Muhammad Azeem}
\date{\today} 
\begin{document} 
\maketitle
\newpage

\section*{Question \# 1 } Consider an $L$-layered feed forward network recursively defined by 

\[ 
h_0 = x, \quad h_{\ell +1} = g(W h_\ell ) , \quad \ell = 1,2, \dots , L-1 , \quad f_W (x) = h_L,
\]
where $x \in \mathbb{R}^d$ is the input vector, $g : \mathbb{R}^d \to \mathbb{R}^d$ is the element-wise activation function, $W \in \mathbb{R}^{d \times d } $  is a shared parameter weight matrix across the network, and $h_\ell$ are the hidden activation at the $\ell$-th layer. 

\subsection*{(a)} Suppose $d=1$, $g$ is the sigmoid activation, and $W = c >0 $ for some positive constant $c \in \mathbb{R}$. Find the tightest possible bounds on the quantity $\displaystyle \abs{\pdv{f_W}{x}}$ and identify what phenomena occurs as we increase the network depth.

\subsection*{Proof.} For, $d=1$ let $g : \mathbb{R}^1 \to \mathbb{R}^1$ be sigmoid function defined as:
\[ 
g(z) = \frac{1}{1+e^{-z}} = \frac{e^z}{1 + e^z}
\]
Furthermore, let $W = c \in \mathbb{R}^+ $ then by recursive definition of $h_\ell $ for $\ell = 1 , 2 , \dots , L-1$ we have:
\[ 
h_L = g( c h_{L-1} ) = g( c (g(c h_{L_1})) ) = \cdots = g ( c ( g(c \cdots g(c h_0)) ) ) 
\]

By chain rule it follows the derivative of $f_W(x)$ with respect of $x$ is 
\[ 
\pdv{f_W}{x} = \pdv{h_L}{x} = \prod_{\ell = 0}^{L-1} \left (  c g^\prime(c h_\ell)  \right) = c^L \prod_{\ell =0}^{L-1} g^\prime (c h_\ell)
\]
Taking the absolute value of above we get:
\begin{align*}
  \abs{\pdv{f_W}{x}}  &= \abs{c^L \prod_{\ell =0}^{L-1} g^\prime (c h_\ell)} = c^L \prod_{\ell = 0}^{L-1} \abs{g^\prime(c h_\ell)}   \tag{$c > 0$}
\end{align*}
Note that since $\displaystyle g(z) = \frac{1}{1+e^{-z}}$ then $e^{-z} = \frac{1}{g(z)} -1$ and the derivative of $g$ using chain rule:

\[ 
g^\prime (z) = \frac{e^{-z}}{(1+e^{-z})^2} =   \left(  \frac{1}{g(z)} -1\right) g^2(z) = g(z)( 1 - g(z) )
\]
Since $0 < g(z) < 1 $, then 
\[ 
\max_{z \in \mathbb{R}} g^\prime (z) = \max_{z \in \mathbb{R}} g(z) ( 1- g(z) ) 
\]
then $g^\prime(z) \leq \frac{1}{4} $ when $g(z) = \frac{1}{2}$ which is at $z=0$ for all $z \in \mathbb{R}$. 
% Therefore, we have an upper bound for $g^\prime (z) \leq \frac{1}{4} $ for all $z \in \mathbb{R}$ 
Then we have

\begin{align*}
  \abs{\pdv{f_W}{x}}  &=  c^L \prod_{\ell = 0}^{L-1} \abs{g^\prime(c h_\ell)}  \\
  &\leq c^L \prod_{\ell = 0}^{L-1} \abs{\frac{1}{4}}   \\
  &= c^L \left(\frac{1}{4}\right)^L = \left(\frac{c}{4}\right)^L
\end{align*}
Thus the tightest bound is 
\[ 
\abs{\pdv{f_W}{x}}  \leq \left(\frac{c}{4}\right)^L
\]
To see what happens as $L$ increases we consider three cases
\begin{enumerate}
\item {If $c<4$:} The factor $\frac{c}{4}$ is less than $1$ so as $L$ increases,
\[ 
   \left(\frac{c}{4}\right)^L \to 0
\]
In other words, the gradients vanish exponentially with the depth of the network. This is the well-known vanishing gradient problem.
\item {If $c>4$:} Then the bound becomes 
\[ 
   \left(\frac{c}{4}\right)^L \to \infty 
\]
as $L$ increases. Thus, the gradients can grow exponentially and explode. This is known as the exploding gradient problem. 
\item {If $c=4$:} Then the bound becomes 
\[ 
   \left(\frac{c}{4}\right)^L = 1^L = 1
\]
In this critical case, the bound is constant, though it is a knife‐edge scenario and actual values may still vary with $x$.

\end{enumerate}

\qed

\subsection*{(b)} Repeat the previous part but use the $\tanh$ activation in place of the sigmoid. Compare your bounds here with what you obtained in the previous part.

\subsection*{Proof.} Let $g : \mathbb{R} \to \mathbb{R}$ defined as 



\end{document}



